\documentclass[a4paper,10pt]{report}

\usepackage[utf8x]{inputenc}
\usepackage[francais]{babel}
\usepackage[T1]{fontenc}
\usepackage{graphicx}
\usepackage{fullpage}

\usepackage[colorlinks=true, urlcolor=blue, linkcolor=black]{hyperref}

\newcommand{\HRule}{\rule{\linewidth}{0.5mm}}

\begin{document}

  % Inclusion de la page de titre
  \input{./dossier_bilan_title.tex}

  % Sommaire
  \tableofcontents

  % Ajout d'une marge entre les paragraphes
  \setlength{\parskip}{0.1in}

  \chapter{Contexte}

  Le projet Fightly a pour objectif la création, dans un premier temps, d'un moteur de jeu Web orienté stratégie au tour par tour, puis dans un deuxième temps de promouvoir ce moteur, notamment en créant des jeux l'utilisant. Ce projet est libre, il devra donc vivre et évoluer par et pour sa communauté. 

  Le but de notre travail, dans le cadre de l'UE TI5 de notre formation, a donc été de lancer ce projet, en en développant les bases et en préparant sa communauté. Ce document présente l'avancement du projet Fightly à la fin de cette première étape, les objectifs atteints, ce qui n'a pas pu être fait, et ce qu'il faudra faire à l'avenir. 

  \chapter{Livrables}

    Le cahier des charges du projet indiquait que nous devions livrer, au terme du projet, trois éléments différents : 

    \begin{quote}
      À la fin du projet, on fournira au client les trois éléments suivants : 

      \begin{list}{-}{}
	\item{Le code source complet de l'application développée ;}
	\item{Une documentation technique, à destination des développeurs qui utiliseront le moteur, mais également à l'intention des développeurs qui contribueront par la suite au projet, celui-ci étant libre ;}
	\item{Une démonstration technique du moteur.}
      \end{list}
    \end{quote}

    Le projet étant libre, l'intégralité du code source est accessible en ligne, via le site de github.com : https://github.com/AdrianGaudebert/fightly-game-engine


  \chapter{Organisation}

    \section{Outils et méthodes}

    \section{Répartition des tâches}

    \section{Concordance avec les prévisions}

  \chapter{Bilan}

    

\end{document}
